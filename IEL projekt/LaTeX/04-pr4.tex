f\section{Příklad 4}
% Jako parametr zadejte skupinu (A-H)
\ctvrtyZadani{D}


Určíme si směry proudů ve smyčkách

\begin{circuitikz}
\draw
    (0,0)
    to (0,2)
    to[resistor, l = $R_1$, *-] (0,4)
    to[sV, v^>=$U_1$] (6,4)
    to (6,2)
    to[open, i_>= $I_{L2}$] (5,2)
    (3,2)
    to [cute inductor, l = $L_2$, v_<=$U_{L2}$, *-*] (6,2)
    to[sV, v^>=$U_2$] (6,0)
    to[capacitor, l = $C_2$, -*] (3,0)
    to[cute inductor, l = $L_1$] (0,0)
    (0,2)
    to[capacitor, l = $C_1$] (3,2)
    to[resistor, l = $R_2$] (3,0)
    (3,3) node[circulator, rotate = 180, scale = 0.5] {}
    (2.5,3) node{$i_A$}
    (1.5,1) node[circulator, rotate = 180, scale = 0.5]{}
    (1,1) node{$i_B$}
    (4.5,0.75) node[circulator, rotate = 180, scale = 0.5]{}
    (4,0.75) node{$i_C$};
\end{circuitikz}

Sestavíme si rovnice proudů ve všech smyčkách
\begin{align*}
    & i_A: R_1 * I_A + Z_{C1} * (I_A - I_B) + Z_{L2} * (I_A - I_C) + U_1 \\
    & i_B: Z_{C1} * (I_B - I_A) + Z_{L1} * I_B + R_2 * (I_B - I_C) \\
    & i_C: R_2 * (I_C - I_B) + U_{C2} * I_C + Z_{L2} * (I_C - I_A) + U_2
\end{align*}

Převedeme rovnice do matic
\begin{equation*}
    \begin{pmatrix}
    R_1 + Z_{C1} + Z_{L2} & -Z_{C1} & -Z_{L2} \\
    -Z_{C1} & Z_{C1} + Z_{L1} + R_1 & - R_2 \\
    -Z_{L2} & -R_2 & R_2 + Z_{C2} + Z_{L2}
    \end{pmatrix}
    *
    \begin{pmatrix}
    I_A \\
    I_B \\
    I_C
    \end{pmatrix}
    =
    \begin{pmatrix}
    -U_1 \\
    0 \\
    -U_2
    \end{pmatrix}
\end{equation*}

Vypočítáme hodnoty impedancí
\begin{align*}
    & \omega = 2 * \pi * f = 2 * \pi * 85 = 534.07075 \\
    & Z_{C1} = -\frac {1} {\omega * C_1}j = -\frac {1} {534.07075111 * 0.00021}j = -8.91624j\\
    & Z_{C2} = -\frac {1} {\omega * C_2}j = -\frac {1} {534.07075111 * 0.000075}j = -24.96548j\\
    & Z_{L1} = \omega * L_1 * j = 534.07075 * 0.18 = 96.13274j \\
    & Z_{L2} = \omega * L_2 * j = 534.07075 * 0.09 = 48.06637j
\end{align*}

Dosadíme hodnoty
\begin{equation*}
    \begin{pmatrix}
        13 - 8.91624j + 48.06637j & 8.91624j & -48.06637j \\
        8.91624j & -8.91624j + 96.13274j + 13 & -15 \\
        -48.06637j & -15 & 15 -24.96548j + 48.06637j
    \end{pmatrix}
    *
    \begin{pmatrix}
        I_A \\
        I_B \\
        I_C
    \end{pmatrix}
    =
    \begin{pmatrix}
        -45 \\
        0 \\
        -50
    \end{pmatrix}
\end{equation*}

Upravíme hodnoty
\begin{equation*}
    \begin{pmatrix}
        13 + 39.15012j & 8.91624j & -48.06637j \\
        8.91624j & 13 + 87.21649j & -15 \\
        -48.06637j & -15 & 15 + 23.10089j
    \end{pmatrix}
    *
    \begin{pmatrix}
        I_A \\
        I_B \\
        I_C
    \end{pmatrix}
    =
    \begin{pmatrix}
        -45 \\
        0 \\
        -50
    \end{pmatrix}
\end{equation*}


Pomocí Sarussova pravidal vypočítáme determinanty matic
\begin{align*}
    \Delta = 
    &\begin{vmatrix}
        13 + 39.15012j & 8.91624j & -48.06637j \\
        8.91624j & 13 + 87.21649j & -15 \\
        -48.06637j & -15 & 15 + 23.10089j
    \end{vmatrix}
    = -71187.16833 + 144197.30794j \\
    \Delta_A = 
    &\begin{vmatrix}
        -45 & 8.91624j & -48.06637j \\
        0 & 13 + 87.21649j & -15 \\
        -50 & -15 & 15 + 23.10089j
    \end{vmatrix}
    =  301624.02492 - 96941.10773j \\
    \Delta_C = 
    &\begin{vmatrix}
        13 + 39.15012j & 8.91624j & -45 \\
        8.91624j & 13 + 87.21649j & 0 \\
        -48.06637j & -15 & -50
    \end{vmatrix}
    =  301624.02492 - 96941.10773j
\end{align*}

Pomocí Cramerova pravidla vypočítáme proudy ve smyčkách A a C
\[ I_A = \frac {\Delta_A} {\Delta} = \frac {301624.02492 - 96941.10773j} {-71187.168332 + 144197.30794j} = -1.37083 - 1.41499j \ A \]
\[ I_C = \frac {\Delta_C} {\Delta} = \frac {346949.95334 - 104238.66133j} {-71187.16833 + 144197.30794j} = -1.53629 - 1.64764j \ A \]

Nyní už jen pomocí vzorečků dopočítáme požadované hodnoty
\begin{align*}
    & I_{L2} = I_A - I_C = -1.37083 - 1.41499j - (-1.53629 - 1.64764j) = 0.165465 + 0.23265j \ A \\[6pt]
    & U_{L2} = I_{L2} * Z_{L2} = (0.165461 + 0.232648j) * 48.06637j = -11.18254 + 7.95313j \ V \\[6pt]
    & |U_{L2}| = \sqrt {Re(U_{L2})^2 + Im(U_{L2})^2} = \sqrt {-11.18254^2 + 7.95313^2} = 13.7223 \ V \\[6pt]
    & \phi_{L2} = arctan\frac {Im(U_{L2})} {Re(U_{L2})} = arctan\frac {7.95313} {-11.18254} + \pi = 2.52338\ rad
\end{align*}


